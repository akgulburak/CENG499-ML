\documentclass{article}
\usepackage[utf8]{inputenc}
\usepackage{array}
\usepackage{wrapfig}
\usepackage{multirow}
\usepackage{tabu}
\usepackage{graphics}

\title{Report}
\author{your name}

\begin{document}

\maketitle
\section{Part 1: Decision Tree}
\subsection{Information Gain}
Test results and referring to the tree diagram.

\subsection{Average Gini Index}
Test results and referring to the tree diagram.

\subsection{Information Gain with Chi-squared Pre-pruning}
Test results and referring to the tree diagram.

\subsection{Average Gini Index with Chi-squared Pre-pruning}
Test results and referring to the tree diagram.


\maketitle
\section{Part 2: Support Vector Machine}
\subsection{First Part}
Draw the resulting classifiers together with data points using draw\_svm function in draw.py and comment on how changing C affects the behavior of the classifier.

\subsection{Second Part}
Draw the resulting classifiers together with data points using draw\_svm function in draw.py and comment on how changing kernels affects the behavior of the classifier.

\subsection{Third Part}
The tables below have the example hyperparameters. Feel free the change them. Report the hyperparameters of your best model and its test accuracy.
\begin{table}[h]
    \centering
    \begin{tabular}{|c|c|c|c|c|c|c|c|}
    \hline
    \multirow{2}{5em}{gamma} & \multicolumn{5}{c|}{C} \\
        & 0.01 & 0.1 & 1 & 10 & 100 \\
        \hline \hline
        -  & val & val & val & val & val \\
        \hline
    \end{tabular}
    \caption{Linear kernel}
    \label{tab:linear}
\end{table}

\begin{table}[h]
    \centering
    \begin{tabular}{|c|c|c|c|c|c|c|c|}
    \hline
    \multirow{2}{5em}{gamma} & \multicolumn{5}{c|}{C} \\
        & 0.01 & 0.1 & 1 & 10 & 100 \\
        \hline \hline
        0.00001  & val & val & val & val & val \\
        0.0001  & val & val & val & val & val \\
        0.001  & val & val & val & val & val \\
        0.01  & val & val & val & val & val \\
        0.1  & val & val & val & val & val \\
        1  & val & val & val & val & val \\
        \hline
    \end{tabular}
    \caption{RBF kernel}
    \label{tab:rbf}
\end{table}

\begin{table}[h]
    \centering
    \begin{tabular}{|c|c|c|c|c|c|c|c|}
    \hline
    \multirow{2}{5em}{gamma} & \multicolumn{5}{c|}{C} \\
        & 0.01 & 0.1 & 1 & 10 & 100 \\
        \hline \hline
        0.00001  & val & val & val & val & val \\
        0.0001  & val & val & val & val & val \\
        0.001  & val & val & val & val & val \\
        0.01  & val & val & val & val & val \\
        0.1  & val & val & val & val & val \\
        1  & val & val & val & val & val \\
        \hline
    \end{tabular}
    \caption{Polynomial kernel}
    \label{tab:poly}
\end{table}

\begin{table}[h]
    \centering
    \begin{tabular}{|c|c|c|c|c|c|c|c|}
    \hline
    \multirow{2}{5em}{gamma} & \multicolumn{5}{c|}{C} \\
        & 0.01 & 0.1 & 1 & 10 & 100 \\
        \hline \hline
        0.00001  & val & val & val & val & val \\
        0.0001  & val & val & val & val & val \\
        0.001  & val & val & val & val & val \\
        0.01  & val & val & val & val & val \\
        0.1  & val & val & val & val & val \\
        1  & val & val & val & val & val \\
        \hline
    \end{tabular}
    \caption{Sigmoid kernel}
    \label{tab:sigmoid}
\end{table}

\newpage
\subsection{Fourth part}
\subsubsection{Without handling the imbalance problem}
Report test accuracy. Can accuracy be a good performance metric?
Report confusion matrix and comment on it. Report additional metrics here if you want.

\subsubsection{Oversampling the minority class}
Report your test accuracy, confusion matrix and comment on them.

\subsubsection{Undersampling the majority class}
Report your test accuracy, confusion matrix and comment on them.

\subsubsection{Setting the class\_weight to balanced}
Report your test accuracy, confusion matrix and comment on them.


\end{document}

